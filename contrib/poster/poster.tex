\documentclass{ctexbeamer}
\usetheme[red]{sjtubeamer}
\usesjtutheme[sepinst]{poster}
% \usesjtutheme[landscape]{poster}
% \usesjtutheme{poster}
\usepackage{lua-visual-debug}
\begin{document}
  \title{poster 子主题}
  \author{作者}
  \logo{\zhlogo}
  \institute[Test Institute]{测试机构}
  \footline[左脚注]{右脚注}
  \begin{frame}[fragile]
    \begin{columns}
      \begin{column}{0.45\textwidth}
        \begin{block}{选项}
          \texttt{poster} 子主题拥有一些选项。
          \begin{description}
            \item[\texttt{sepinst}] 分离徽标与机构于两侧
            \item[\texttt{landscape}] 横向海报 
          \end{description}
        \end{block}
        \begin{alertblock}{脚注}
          使用 \texttt{\textbackslash{}section} 和 \texttt{\textbackslash{}subsection} 设定左脚注和右脚注。
        \end{alertblock}
        \begin{exampleblock}{区块}
          仍然可以使用内置的 \texttt{block}, \texttt{alertblock}, \texttt{exampleblock} 插入对应的区块。
        \end{exampleblock}
      \end{column}
      \begin{column}{0.45\textwidth}
        
        \begin{stampblock}{印记区块}
          使用 \texttt{stampblock} 环境插入带有印记图案的区块,编号会自动递增。
        \end{stampblock}
        
        
        \begin{codeblock}[escapechar=|]{代码块}
% 应当减少代码块的使用,增加了标题栏的高度。
        \end{codeblock}

        \begin{columns}[b]
          \begin{column}{.5\textwidth}
            \begin{stampbox}
              \includegraphics[width=16cm]{sjtuphoto.jpg}
            \end{stampbox}
          \end{column}
          \begin{column}{.5\textwidth}
            $\leftarrow$ 你仍然可以使用 \texttt{stampbox} 环境插入带边框的图片\vspace{1ex}
          \end{column}
        \end{columns}
      \end{column}
    \end{columns}
  \end{frame}
\end{document}