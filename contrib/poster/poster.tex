\documentclass{ctexbeamer}
\usetheme{sjtubeamer}
\usesjtutheme[sepinst]{poster}
% \usesjtutheme[landscape]{poster}
% \usesjtutheme{poster}
\usepackage{lua-visual-debug}
\begin{document}
    \title{poster 子主题}
    \author{作者}
    \institute[Test Institute]{测试机构}
    \section{左脚注}
    \subsection{右脚注}
    \begin{frame}[fragile]
        \codeblockinput[firstnumber=3,firstline=3,lastline=3]{开始使用}{poster.tex}
        \texttt{sepinst} 选项用于指定是否将机构部分分离放置。
        \codeblockinput[firstnumber=7,firstline=7,lastline=7]{使用机构名称}{poster.tex}
        在 \texttt{ctexbeamer} 文档类中,机构将采用中文样式,需要指定可选的机构英文全称。在 \texttt{beamer} 文档类中,将会直接使用可选参数的英文全称。如果需要使用图片作为机构名称,需要使用 0.95cm 高度的图标以居中显示,并放置在会被显示的参数上。
        
        \begin{codeblock}[]{中文组合徽标}
% \documentclass{ctexbeamer}
\institute[]{\resizebox{!}{0.95cm}{\zhlogo}}
        \end{codeblock}

        \secondaryinstlogo[]{\resizebox{!}{0.95cm}{\zhlogo}}{\sjtubadge}
        
        \begin{codeblock}[]{英文组合徽标}
% \documentclass{beamer}
\institute[\resizebox{!}{0.95cm}{\enlogo}]{}
        \end{codeblock}

        \secondaryinstlogo{\resizebox{!}{0.95cm}{\enlogo}}{\sjtubadge}
    \end{frame}
\end{document}