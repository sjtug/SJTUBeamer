\documentclass[
  aspectratio=169,
  fontset=ubuntu, % 使用 Noto Sans CJK 字体,需要先安装字体
]{ctexbeamer}
\usetheme[
  max, % 可以替换成 maxplus, min 等以更换封面
  red,
  light, % 可以更换成 dark 使用深色主题
]{sjtubeamer}
\usesjtutheme{touying} % 使用 touying 子主题

% \usepackage{lua-visual-debug}

\usepackage{fontawesome5}

\begin{document}
\title{touying子主题}
\subtitle{仿touying-sjtu的上海交通大学幻灯片模板}
\author{作者}
\institute[SJTU]{上海交通大学}
\maketitle
\logo{} % 清空每页的 logo

\part{第一部分}

\begin{frame}{目录}
\tableofcontents
\end{frame}

\section{第一小节}

\subsection{第一小小节}

\begin{frame}[t]{主题说明}
本主题仿照 touying-sjtu \href{https://github.com/sjtug/touying-sjtu}{\faLink} 制作。在 \texttt{frame} 环境上添加 \texttt{[t]} 选项可以让内容顶端对齐,这样就和原作比较像了。
\end{frame}

\subsection{第二小小节}

\begin{frame}{标题}
\framesubtitle{子标题}
这一页是含有子标题的幻灯片。$a+b=c$
\end{frame}

\section{第二小节}
\begin{frame}
这一页没有标题的幻灯片。
\end{frame}

\section{第三小节}

\makebottom

\end{document}