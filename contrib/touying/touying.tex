\documentclass[
  aspectratio=169,
  fontset=ubuntu, % 使用 Noto Sans CJK SC 字体,需要先安装字体
]{ctexbeamer}
\usetheme[
  max, % 可以替换成 maxplus, min 等以更换封面
  red, % 可以更换为 blue 颜色主题
  light, % 可以更换成 dark 使用深色主题
]{sjtubeamer}
\usesjtutheme{touying} % 使用 touying 子主题

\usepackage{zhlipsum}
\usepackage{fontawesome5}

\begin{document}
\title{touying子主题}
\subtitle{仿touying-sjtu的上海交通大学幻灯片模板}
\author{作者}
\institute[SJTU]{上海交通大学}
\date{2026-01-27}
\maketitle

\begin{frame}{目录}
  \tableofcontents
\end{frame}

\section{第一小节}

\subsection{第一小小节}

\begin{frame}[t]{主题说明}
  本主题仿照 touying-sjtu \href{https://github.com/sjtug/touying-sjtu}{\faLink} 制作。

  这里使用了 Noto Sans CJK SC 及 Libertinus 系列字体,编译前需要安装相关字体。

  在 \texttt{frame} 环境上添加 \texttt{[t]} 选项可以让内容顶端对齐,这样就和原作比较像了。

  封面可以使用 \texttt{maxplus}/\texttt{max}/\texttt{min} 切换,颜色主题可以使用 \texttt{red}/\texttt{blue} 切换,明暗模式可以使用 \texttt{light}/\texttt{dark} 切换。
\end{frame}

\subsection{第二小小节}

\begin{frame}[t]
  \frametitle{标题}
  \framesubtitle{子标题}
  这一页是含有子标题的幻灯片。$a+b=c$
\end{frame}

\section{第二小节}
\begin{frame}
  这一页是没有标题居中的幻灯片。\footnote{这是一个脚注。}
\end{frame}

\section{第三小节}

\begin{frame}[t,allowframebreaks] % 允许内容跨页
  \frametitle{内容跨页}
  \zhlipsum[1-2]
\end{frame}

\makebottom

\end{document}