\documentclass[
    UTF8,
    heading=true,
    12pt
]{ctexrep}

\pagestyle{plain}

\usepackage{pifont}

\ctexset {
    part = {
        format = \bfseries\Huge,
        number = ,
        name = \themename,
        nameformat = \huge
    },
    chapter = {
        beforeskip = 0pt,
        fixskip = true,
        format = \bfseries,
        name = ,
        nameformat = \hfill\huge,
        number = \arabic{chapter},
        aftername = \,\par\hfill,
        titleformat = \LARGE,
        aftertitle = \par\bigskip\nointerlineskip\rule{\linewidth}{2bp}\par
    },
    section = {
        format = \bfseries\large\raggedright,
        number = \raisebox{-.1ex}{\ding{\numexpr181 + \arabic{section}}},
        numberformat = \Large
    }
}

\usepackage[colorlinks]{hyperref}

\usepackage{enumitem}
\setlist[enumerate]{
    itemsep=0pt,
    left=\parindent,
    label=\raisebox{-.1ex}{\large\ding{\numexpr171 + \arabic*}}
}

\usepackage{tcolorbox}
\newtcbox{\xbutton}[1][red]{on line,
arc=3pt,colback=#1!50!black,colframe=#1!50!black,
before upper={\rule[-3pt]{0pt}{10pt}},boxrule=1pt,
boxsep=0pt,left=6pt,right=6pt,top=2pt,bottom=0pt,
fontupper=\sffamily,colupper=white}

\usepackage{iftex}
\ifpdftex%
\else
    \usepackage{fontspec}
\fi
\usepackage{fontawesome}
\usepackage{hologo}

\lstset{
    language=[LaTeX]TeX,
    basicstyle=\ttfamily,
    columns=flexible, 
    extendedchars=false
}

\def\themename{\textsf{SJTUBeamer}}

\begin{document}
    \title{\themename}
    \date{\today}
    % \maketitle

    \chapter*{\themename\ 介绍}

    \themename\ 是基于 \verb"beamer" 文档类的主题宏集。

    \part{基础操作}
    \chapter{安装}

    \section{运行环境}

    为了使用 \themename\ 的全部功能,请使用下者之一:
    \begin{enumerate}
        \item \href{https://miktex.org/}{MiK\TeX\ }
        \item \href{https://mirrors.sjtug.sjtu.edu.cn/ctan/systems/texlive/Images/texlive2021-20210325.iso}{\TeX\ Live} 2020 及以上版本
    \end{enumerate}

    % 上传至 Overleaf 后添加对应在线步骤
    % 上传至 CTAN 后本段可以直接更改为安装宏包的标准流程
    \section{下载模板}

    \begin{enumerate}
        \item 前往 GitHub 上的 \faGithub{}~\href{https://github.com/sjtug/SJTUBeamer}{sjtug/SJTUBeamer} 页面,点击 \xbutton[green]{Code} 按钮下载压缩文件,在解压后的主目录里新建 \TeX{} 源文件即可调用该模板。
        \item 或者点击进入存储库侧栏的 \href{https://github.com/sjtug/SJTUBeamer/releases}{\textsf{Releases}} 下载查看最新发布版本,并下载 \textsf{Assets} 栏的 \texttt{sjtubeamer-ctan.zip}。
    \end{enumerate}
    
    \section{测试模板}

    \begin{enumerate}
        \item[\faWindows] Windows 系统采用 \hologo{pdfLaTeX} 编译以获得更快的编译速度。
        \item[\faApple\ \faLinux] *nix 系统采用 \hologo{XeLaTeX} 编译以获得中文支持。
    \end{enumerate}

    

    \part{进阶操作}
    \chapter{修改}

    \part{高级操作}
    \chapter{源代码}

\end{document}