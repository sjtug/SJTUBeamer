\documentclass[
    % draft,                             % 草稿模式
    aspectratio=169,                   % 使用 16:9 比例
]{ctexbeamer}
\mode<presentation>
\usepackage{pgfplots}
\usepackage{pgfplotstable}
\usetheme[min]{sjtubeamer}
% \usecolortheme[]{beaver}                 % 使用其他颜色主题
\usepackage{biblatex}
\addbibresource{ref.bib}               % gbt!=bibtex
\usepackage{multicol}

\begin{document}
\institute[School of Mathematical Sciences]{数学科学学院}   % 组织
% \logo{
%     \includegraphics{vi/zhlogored.pdf}  % 重定义 logo
% }
\titlegraphic{                         % 标题图像
  \includegraphics{head.png}
}
\title{SJTUBeamer 幻灯片模板}  % 标题
\subtitle{SJTUBeamer Template}         % 副标题
\author{Anxue Chen, Alexara Wu, Log Creative}                  % 作者
\date{\today}                          % 日期  
\maketitle                             % 创建标题页

\part{第一部分}

% 使用节目录
% \AtBeginSection[]{
%     \begin{frame}
%         % \tableofcontents[currentsection]           % 传统节目录             
%         \sectionpage                   % 节页
%     \end{frame}
% }

% 使用小节目录
\AtBeginSubsection[]{                  % 在每小节开始
  \begin{frame}
    % \tableofcontents[currentsection,currentsubsection]             % 传统小节目录             
    \subsectionpage                % 小节页
  \end{frame}
}

\section{第 1 节}
\subsection{第 1 小节}

\begin{frame}
  \frametitle{标题}

  \paragraph{列表} 这个\alert{幻灯片}有下面几项:

  \begin{itemize}
    \item 第 1 项
    \item 第 2 项
    \item 第 3 项
  \end{itemize}

\end{frame}

\begin{frame}
  \frametitle{标题}
  \framesubtitle{子标题}

  \begin{equation}
    x^2+2x+1=(x+1)^2
  \end{equation}

\end{frame}

\section{第 2 节}
\begin{frame}
  \frametitle{一些盒子}

  \begin{block}{盒子}
    这是一个盒子\cite{beamerman}
  \end{block}

  \begin{alertblock}{注意}
    注意内容
  \end{alertblock}

  \begin{exampleblock}{示例}
    示例内容
  \end{exampleblock}
\end{frame}

\begin{frame}[fragile]          % 注意添加 fragile 标记
  \frametitle{代码块}
  % 代码块参数:语言,标题
  % 请减少代码初始的缩进
  \begin{codeblock}[language=c++]{C++代码}
    #include<iostream>

    int main(){
        // Console Output
        std::cout << "Hello, SJTU!" << std::endl;
        return 0;
      }
  \end{codeblock}
\end{frame}

\begin{frame}
  \frametitle{图}
  \begin{figure}
    \centering
    \begin{stampbox}
      \includegraphics[height=0.3\textheight]{plant.jpg}
    \end{stampbox}
    \caption{图片标题\cite{viman}}
  \end{figure}
\end{frame}

\begin{frame}
  \frametitle{表与统计图}
  \begin{multicols}{2}
    \begin{table}
      \caption{表格标题\cite{pgfplotstableman}}
      \pgfplotstabletypeset[
        columns/Quick/.style={dec sep align},
        columns/Cocktail/.style={dec sep align},
        column type=r,
        % fixed zerofill,
      ]{test.csv}
    \end{table}

    \begin{figure}
      \input{testgraph.tex}
      \caption{统计图标题\cite{pgfplotsman}}
    \end{figure}
  \end{multicols}
\end{frame}


% gbt=bibtex
\part{参考文献}
\begin{frame}[allowframebreaks]
  \printbibliography[title=参考文献]    % gbt!=bibtex
  % \bibliography{ref.bib}             % gbt=bibtex
\end{frame}

\makebottom     % 创建尾页  % 非标准命令

\end{document}