\documentclass[
    % draft,                             % 草稿模式
    aspectratio=169,                   % 使用 16:9 比例
]{beamer}
\mode<presentation>
\usetheme[
    navigation=subsections,            % 使用子章节进度显示
    % lang=en,                           % 使用英文logo
    % cjk=true,                          % 使用CJK而不是ctex
    color=red,                         % 使用红色主题
    % pattern=none,                      % 不使用图案装饰
    % biber=false,                       % 使用默认的 bibtex 引擎
]{SJTUBeamer}
\institute[School of Mathematical Sciences]{数学科学学院}   % 组织
\title{\textsf{SJTUBeamer} 幻灯片模板}  % 标题
\subtitle{SJTUBeamer Template}         % 副标题
\author{Log Creative}                  % 作者
\date{\today}                          % 日期
\addbibresource{ref.bib}  % 推荐使用 JabRef 对文献进行管理

\begin{document}

    \maketitle                         % 创建标题页

\part{第一部分}

    \begin{frame}
        \frametitle{提纲}
        \tableofcontents               % 创建目录
    \end{frame}

\section{第 1 节}
\subsection{第 1 小节}

    \begin{frame}
        \frametitle{标题}

        \begin{itemize}
            \item 第 1 项
            \item 第 2 项
            \item 第 3 项
        \end{itemize}

    \end{frame}

    \begin{frame}
        \frametitle{标题}
        \framesubtitle{子标题}

        \begin{equation}
            x^2+2x+1=(x+1)^2
        \end{equation}
        
    \end{frame}

\section{第 2 节}
    \begin{frame}
        \frametitle{一些盒子}
        
        \begin{block}{盒子}
            这是一个盒子\cite{beamerman}
        \end{block}

        \begin{alertblock}{注意}
            注意内容
        \end{alertblock}

        \begin{exampleblock}{示例}
            示例内容
        \end{exampleblock}
    \end{frame}

    \begin{frame}[fragile]          % 注意添加 fragile 标记
        \frametitle{代码块}
        % 代码块参数:语言,标题
        % 请减少代码初始的缩进
        \begin{codeblock}{c++}{C++代码}
#include<iostream>

int main(){
    // Console Output
    std::cout << "Hello, SJTU!" << std::endl;
    return 0;
}
        \end{codeblock}
    \end{frame}

    \begin{frame}
        \frametitle{图}
        \begin{figure}
            \centering
            \begin{stampbox}
                \includegraphics[height=0.3\textheight]{vi/plant.jpg}
            \end{stampbox}
            \caption{图片标题\cite{viman}}
        \end{figure}
    \end{frame}

    \begin{frame}
        \frametitle{表与统计图}
        \begin{multicols}{2}
        \begin{table}
            \caption{表格标题\cite{pgfplotstableman}}
            \pgfplotstabletypeset[
                columns/Quick/.style={dec sep align},
                columns/Cocktail/.style={dec sep align},
                column type=r,
                % fixed zerofill,
            ]{dat/test.csv}
        \end{table}
        
        \begin{figure}
            \input{img/testgraph.tex}
            \caption{统计图标题\cite{pgfplotsman}}
        \end{figure}
        \end{multicols}
    \end{frame}

\appendix

    \begin{frame}
        \frametitle{参考文献}
        \printbibliography[title=参考文献]
    \end{frame}

    \makebottom     % 创建尾页  % 非标准命令

\end{document}