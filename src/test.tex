\documentclass{beamer}
\mode<presentation>
\usetheme[
    % navigation=subsections,            % 使用子章节进度导航
    % lang=en,                           % 使用英文logo
    % cjk=true,                          % 使用CJK而不是ctex,上一个选项将会将此覆盖
    % color=red,                         % 使用红色主题
]{SJTUBeamer}

\title{\textsf{SJTUBeamer} 幻灯片模板}  % 标题
\subtitle{SJTUBeamer Template}         % 副标题
\author{Log Creative}                  % 作者
\institute[]{github.com/LogCreative}   % 组织
\date{\today}                          % 日期

\begin{document}

    \maketitle                         % 创建标题页

    \begin{frame}
        \frametitle{提纲}
        \tableofcontents               % 创建目录
    \end{frame}

    \section{第 1 节}
    \subsection{第 1 小节}
    \begin{frame}
        \frametitle{标题}

        \begin{itemize}
            \item 第 1 项
            \item 第 2 项
            \item 第 3 项
        \end{itemize}

    \end{frame}

    \begin{frame}
        \frametitle{标题}
        \framesubtitle{子标题}

        \begin{equation}
            x^2+2x+1=(x+1)^2
        \end{equation}
        
    \end{frame}

    \section{第 2 节}
    \begin{frame}
        \frametitle{一些盒子}
        
        \begin{block}{盒子}
            这是一个盒子
        \end{block}

        \begin{alertblock}{注意}
            注意内容
        \end{alertblock}

        \begin{exampleblock}{示例}
            示例内容
        \end{exampleblock}
    \end{frame}

    \begin{frame}[fragile]          % 注意添加 fragile 标记
        \frametitle{代码块}
        % 代码块参数:语言,标题
        % 请减少代码初始的缩进
        \begin{codeblock}{c++}{C++代码}
#include<stdio.h>

int main(){
    // Console Output
    std::cout << "Hello, SJTU!" << std::endl;
    return 0;
}
        \end{codeblock}
    \end{frame}

\end{document}