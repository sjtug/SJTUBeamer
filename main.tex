% !TeX encoding = UTF-8

\documentclass{beamer}

\usepackage[utf8]{inputenc}
\usepackage{listings}
\usepackage{graphicx}
\usepackage{hyperref}
\usepackage{ctex}
\usepackage{amsmath}


\usetheme{Szeged}
\usecolortheme{beaver}
\usefonttheme{structurebold}
\useinnertheme{circles}
\setbeamertemplate{blocks}[rounded]
\definecolor{blocktitlecolor}{rgb}{0.8 0 0}
\setbeamercolor{block title}{fg=white,bg=blocktitlecolor}
\setbeamertemplate{background}{\includegraphics[width=\paperwidth]{sjtu-miao.png}}
\setbeamercolor{frametitle}{bg=}


%------------------------------------------------------------
%This block of code defines the information to appear in the
%Title page
\setbeamercolor{title}{bg=}
\title[About Beamer] %optional
{\textbf{About the Beamer class in presentation making}}


% \subtitle{A short story}

% \author[Arthur, Doe] % (optional)
% {A.~B.~Arthur\inst{1} \and J.~Doe\inst{2}}
\author[Anxue Chen]
{Anxue Chen}

% \institute[VFU] % (optional)
% {
%   \inst{1}%
%   Faculty of Physics\\
%   Very Famous University
%   \and
%   \inst{2}%
%   Faculty of Chemistry\\
%   Very Famous University
% }
\institute[SJTU]{Shanghai Jiao Tong University}

% \date[VLC 2014] % (optional)
% {Very Large Conference, April 2014}
\date{April 2021}

\logo{\includegraphics[height=1.5cm]{vi/sjtu-vi-badge-red.pdf}}

%End of title page configuration block
%------------------------------------------------------------



%------------------------------------------------------------
%The next block of commands puts the table of contents at the 
%beginning of each section and highlights the current section:
%\AtBeginSection for each subsection

\AtBeginSection[]
{
  \begin{frame}
    \frametitle{Table of Contents}
    \tableofcontents[currentsection]
  \end{frame}
}
%------------------------------------------------------------


\begin{document}

%The next statement creates the title page.
\frame{\titlepage}


%---------------------------------------------------------
%This block of code is for the table of contents after
%the title page
\begin{frame}
	\frametitle{Table of Contents}
	\tableofcontents
\end{frame}
%---------------------------------------------------------


\section{First section}

%---------------------------------------------------------
%Changing visivility of the text
\begin{frame}
	\frametitle{Sample frame title}
	This is a text in second frame. For the sake of showing an example.

	\begin{itemize}
		\item<1-> Text visible on slide 1
		\item<2-> Text visible on slide 2
		\item<3> Text visible on slides 3
		\item<4-> Text visible on slide 4
	\end{itemize}
\end{frame}

%---------------------------------------------------------


%---------------------------------------------------------
%Example of the \pause command
\begin{frame}
	In this slide \pause

	the text will be partially visible \pause

	And finally everything will be there
\end{frame}
%---------------------------------------------------------

\section{Second section}

%---------------------------------------------------------
%Highlighting text
\begin{frame}
	\frametitle{Sample frame title}

	In this slide, some important text will be
	\alert{highlighted} because it's important.
	Please, don't abuse it.

	\begin{block}{Remark}
		Sample text
	\end{block}

	\begin{alertblock}{Important theorem}
		Sample text in red box
	\end{alertblock}

	\begin{examples}
		Sample text in green box. The title of the block is ``Examples".
	\end{examples}
\end{frame}
%---------------------------------------------------------


%---------------------------------------------------------
%Two columns
\begin{frame}
	\frametitle{Two-column slide}

	\begin{columns}

		\column{0.5\textwidth}
		This is a text in first column.
		$$E=mc^2$$
		\begin{itemize}
			\item First item
			\item Second item
		\end{itemize}

		\column{0.5\textwidth}
		This text will be in the second column
		and on a second tought this is a nice looking
		layout in some cases.
	\end{columns}
\end{frame}
%---------------------------------------------------------

\section{Third section}
% \begin{frame}[fragile]
% \frametitle{Code}
%   \begin{lstlisting}[language=C,tabsize=2]
%   #include <stdio.h>
%   #include <unistd.h>
%   #include <sys/types.h>
%   #include <sys/wait.h>

%   // This is a comment
%   int main(int argc, char **argv)
%   {
%           while (--c > 1 && !fork());
%           sleep(c = atoi(v[c]));
%           printf("%d\n", c);
%           wait(0);
%           return 0;
%   }
%     \end{lstlisting}
% \end{frame}

% \begin{block}{Block 1}
% This is Block 1.
% \end{block}

% \begin{figure}
%     \centering
%     \includegraphics[height=4cm]{logo.png}
%     \caption{Caption}
%     \label{fig:my_label}
% \end{figure}

%---------------------------------------------------------

\end{document}
